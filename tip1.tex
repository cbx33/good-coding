\chapter{Tip 1 - First things first}
\subsection{(or Why didn't you tell me that in the first place)}

\textit{Primo fac prima}. It's Latin and it means; First things first. 
Though we often all have the intention of doing things in the right order, we don't always get it right.
Getting it wrong can lead to a lot of wasted time and often it's not just your own time that you're wasting, it's other people's time too.
Consider the following conversation as a case in point;

\begin{trenches}
Alvin: I've got an issue with my computer.

Bryan: What seems to be wrong?

Alvin: It doesn't print when I ask it to.

Bryan: Have you tried turning the printer off and on again?

Alvin: Yes.

Bryan: Have you reinstalled the print drivers?

Alvin: Yes.

Bryan: Can you hook it up to another computer and try it?

Alvin: No, I've run out of ink.
\end{trenches}

It's down right infuriating isn't it and it happens a lot.
Doing things in the right order is essential and dealing with information in the right order is essential too.
When you are evaluating a set of criteria during a selection process you will usually order them according to which is the most important.
Why do we do this? To save time and effort. 
If you \textbf{must} have 3 bedrooms in a house, you don't consider one bedroom apartments just because they have en suite bathrooms.
Why? Because 3 bedrooms is your top priority, it is what you care about the most.

In the land of software, ordering your conditional statements can be a tricky and subtle business, but let us examine the piece of code below.

\begin{code}
def prime(number):
	for test in range(number/2+1)[2:]:
		if number%test == 0:
			return 0
	return 1

def even(number):
	if number%2 == 0:
		return 1
	else:
		return 0

number_list = range(1,50000)

def slow_function():
	special_numbers = []
	for number in number_list:
		print "Trying: ", number
		if prime(number):
			if number < 1000:
				special_numbers.append(number)
		elif even(number):
			if number < 1000:
				special_numbers.append(number)

def fast_function():
	special_numbers = []
	for number in number_list:
		print "Trying: ", number
		if number < 1000:
			if prime(number):
				special_numbers.append(number)
			elif even(number):
				special_numbers.append(number)
	print special_numbers

slow_function() #14 seconds
fast_function() #1 second
\end{code}

\begin{callout}{Remember}{You don't get nothing for free}
There is always some compromise
\end{callout}
