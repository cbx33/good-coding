\chapter{Tip 1 - First things first}
\subsection{(or Why didn't you tell me that in the first place)}

\textit{Primo fac prima}. It's Latin and it means; First things first. 
Though we often all have the intention of doing things in the right order, we don't always get it right.
Getting it wrong can lead to a lot of wasted time and often it's not just your own time that you're wasting, it's other people's time too.
Consider the following conversation as a case in point;

\begin{trenches}
Alvin: I've got an issue with my computer.

Bryan: What seems to be wrong?

Alvin: It doesn't print when I ask it to.

Bryan: Have you tried turning the printer off and on again?

Alvin: Yes.

Bryan: Have you reinstalled the print drivers?

Alvin: Yes.

Bryan: Can you hook it up to another computer and try it?

Alvin: No, I've run out of ink.
\end{trenches}

It's down right infuriating isn't it and it happens a lot.
Doing things in the right order is essential and dealing with information in the right order is essential too.
When you are evaluating a set of criteria during a selection process you will usually order them according to which is the most important.
Why do we do this? To save time and effort.

If you \textbf{must} have 3 bedrooms in a house, you don't consider one bedroom apartments just because they have en suite bathrooms.
Why? Because 3 bedrooms is your top priority. It is what you care about the most.

In the land of software, ordering your conditional statements can be a tricky and subtle business, but let us examine the piece of code below.

\begin{code}
def prime(number):
	for test in range(number/2+1)[2:]:
		if number%test == 0:
			return 0
	return 1

def even(number):
	if number%2 == 0:
		return 1
	else:
		return 0

number_list = range(1,50000)

def function():
	special_numbers = []
	for number in number_list:
		print "Trying: ", number
		if prime(number):
			if number < 1000:
				special_numbers.append(number)
		elif even(number):
			if number < 1000:
				special_numbers.append(number)
	print special_numbers
function()
\end{code}

Now, let us ignore how the functions \texttt{prime()} and \texttt{even()} actually work.
We will assume for now, that they return either \texttt{True} or \texttt{False}, to indicate that the number supplied is either \textit{prime} or \textit{even}.
So, we create an array of numbers from 1 to 50,000 and call it number\_list.
We then define a function, which steps through these numbers one by one and is intended to create a list of \textit{even} and \textit{prime} numbers.

As you can see, we have our conditionals in the code to check this. There is also another check made to see if the number is under 1,000.
There is nothing wrong with this code. It works perfectly well. On my machine it takes a little over 14 seconds to run.
Can you see a potential area for improvement?

Of course! The most important criteria to us, the one that has the biggest impact, is that the number should be under 1,000. 
That rules out most of the numbers. 
If we shift around the conditionals slightly, we can make the code run much faster. Consider the code below.

\begin{code}
def function():
	special_numbers = []
	for number in number_list:
		print "Trying: ", number
		if number < 1000:
			if prime(number):
				special_numbers.append(number)
			elif even(number):
				special_numbers.append(number)
	print special_numbers
\end{code}

Notice now that we have managed to weed out most of the numbers at the first hurdle.
As it turns out, the function for testing for prime is actually quite costly.
We will ignore the fact that this function in itself could be improved.

%\begin{callout}{Remember}{You don't get nothing for free}
%There is always some compromise
%\end{callout}
